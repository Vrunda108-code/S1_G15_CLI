\documentclass{article}
\usepackage{amsmath, amssymb}
\usepackage{geometry}
\geometry{margin=1in}

\begin{document}

\section*{CSE 400: Fundamentals of Probability in Computing}

\subsection*{Lecture 8 – Continuous Random Variables and Gaussian Random Variable}

\hrule
\vspace{1em}

\section*{1. Continuous Random Variable (CRV)}

\subsection*{Definition}

A random variable (X) is a \textbf{continuous random variable} if it is characterized by a \textbf{probability density function (PDF)} $(f_X(x))$.

\subsection*{Properties of PDF}

\begin{itemize}
    \item $(f_X(x) \ge 0)$ for all $(x)$
    \item
\end{itemize}

\[
\int_{-\infty}^{\infty} f_X(x),dx = 1
\]

\subsection*{Probability from PDF}

For any interval $([x_1, x_2])$:
\[
P(x_1 \le X \le x_2) = \int_{x_1}^{x_2} f_X(x),dx
\]

\hrule
\vspace{1em}

\section*{2. Cumulative Distribution Function (CDF)}

\subsection*{Definition}

The \textbf{CDF} of a continuous random variable (X) is defined as:
\[
F_X(x) = P(X \le x) = \int_{-\infty}^{x} f_X(t),dt
\]

\subsection*{Interval Probability Using CDF}

\[
P(x_1 \le X \le x_2) = F_X(x_2) - F_X(x_1)
\]

\hrule
\vspace{1em}

\section*{3. Expectation of a Continuous Random Variable}

\subsection*{Expectation of a Function of X}

For a function $(g(X))$:
\[
E[g(X)] = \int_{-\infty}^{\infty} g(x) f_X(x),dx
\]

\subsection*{Mean (Expected Value)}

\[
\mu_X = E[X] = \int_{-\infty}^{\infty} x f_X(x),dx
\]

\hrule
\vspace{1em}

\section*{4. Moments of a Random Variable}

\subsection*{n-th Order Moment (About the Origin)}

\[
E[X^n]
\]

Special cases:

\begin{itemize}
    \item $(n = 0)$: $(E[X^0] = E[1] = 1)$
    \item $(n = 1)$: $(E[X] = \mu_X)$
    \item $(n = 2)$: $(E[X^2])$
\end{itemize}

\hrule
\vspace{1em}

\section*{5. Central Moments}

\subsection*{Definition}

The \textbf{n-th order central moment} is defined as:
\[
E[(X - \mu_X)^n]
\]

\subsection*{Special Cases}

\begin{itemize}
    \item $(n = 0)$:
    \[
    E[(X - \mu_X)^0] = E[1] = 1
    \]

    \item $(n = 1)$:
    \[
    E[(X - \mu_X)] = E[X] - \mu_X = 0
    \]

    \item $(n = 2)$: Variance
\end{itemize}

\hrule
\vspace{1em}

\section*{6. Variance}

\subsection*{Definition}

\[
\sigma_X^2 = E[(X - \mu_X)^2]
\]

\subsection*{Derivation}

\[
E[(X - \mu_X)^2]
= E[X^2 - 2\mu_X X + \mu_X^2]
\]

Using linearity of expectation:
\[
= E[X^2] - 2\mu_X E[X] + \mu_X^2
\]

Since $(E[X] = \mu_X)$:
\[
\sigma_X^2 = E[X^2] - \mu_X^2
\]

\hrule
\vspace{1em}

\section*{7. Moments vs Central Moments}

\begin{itemize}
    \item Moments: $(E[X^n])$
    \item Central Moments: $(E[(X - \mu_X)^n])$
\end{itemize}

Key distinction:
\[
E[X]^2 \neq E[X^2]
\]

\hrule
\vspace{1em}

\section*{8. Skewness}

\subsection*{Definition (3rd Central Moment)}

\[
C_s = \frac{E[(X - \mu_X)^3]}{\sigma_X^3}
\]

\subsection*{Interpretation}

\begin{itemize}
    \item $(C_s > 0)$: Right-skewed PDF
    \item $(C_s < 0)$: Left-skewed PDF
\end{itemize}

\hrule
\vspace{1em}

\section*{9. Kurtosis}

\subsection*{Definition (4th Central Moment)}

\[
C_k = E[(X - \mu_X)^4]
\]

\subsection*{Observation}

\begin{itemize}
    \item Large kurtosis indicates a large peak near the mean
\end{itemize}

\hrule
\vspace{1em}

\section*{10. Linearity of Expectation}

\subsection*{Theorem}

For constants $(a)$ and $(b)$:
\[
E[aX + b] = aE[X] + b
\]

\subsection*{Extension}

If $(g(x) = g_1(x) + g_2(x) + \cdots + g_n(x))$, then:
\[
E[g(X)] = E[g_1(X)] + E[g_2(X)] + \cdots + E[g_n(X)]
\]

\hrule
\vspace{1em}

\section*{11. Gaussian (Normal) Random Variable}

\subsection*{Definition}

A random variable (X) is Gaussian if:
\[
X \sim \mathcal{N}(\mu_X, \sigma_X^2)
\]

\subsection*{Probability Density Function}

\[
f_X(x) = \frac{1}{\sqrt{2\pi},\sigma_X}
\exp\left(-\frac{(x - \mu_X)^2}{2\sigma_X^2}\right)
\]

\hrule
\vspace{1em}

\section*{12. Standard Normal Distribution}

\subsection*{Definition}

If:
\[
\mu_X = 0, \quad \sigma_X^2 = 1
\]

Then (X) follows the \textbf{standard normal distribution}.

\hrule
\vspace{1em}

\section*{13. Symmetry of Gaussian Distribution}

\begin{itemize}
    \item PDF is symmetric about $(x = \mu_X)$
    \item Left and right areas around the mean are equal
\end{itemize}

\hrule
\vspace{1em}

\section*{14. Worked Probability Example (from Lecture)}

Given:
\[
|X + 3| < 2
\]

Step-by-step:
\[
-2 < X + 3 < 2
\]
\[
-5 < X < -1
\]

Probability:
\[
P(-5 < X < -1) = F_X(-1) - F_X(-5)
\]

\hrule
\vspace{1em}

\section*{End of Lecture 8 Scribe}

\end{document}
